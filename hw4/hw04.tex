\documentclass[]{article}

\title{Homework 4}
\author{Patrick Landis (pal25)}
\date{02/23/2013}
\usepackage[margin=1in]{geometry}
\usepackage{multicol}

\begin{document}
\maketitle

\section{H2O Bonding with Monitors}
\subsection{Monitor}
\begin{verbatim}
type BondActivity = monitor
    var counts: int[2]
    var ready: condition[2]

    // Procedure to increment the count of hydrogen or oxygen
    procedure entry incCount(int i)
    begin
        counts[i]++
    end
    
    // Check if we are ready for bonding
    procedure entry isReady(int i)
    begin
        if counts[0] >= 2 and counts[1] >=1
            ready[0].signal
   	        ready[(i+1)%2].signal // Signals the other process not currently in the monitor.
   	        counts[0] -= 2
   	        counts[1] -= 1
        else
            ready[i].wait
        endif
    end
	
    // This procedure is here to ensure that
    // a process isnt scheduled out between
    // incrementing the count and checking
    // if a bond can happen.
    procedure entry incAndCheck(int i)
        incCount(i)
        isReady(i)
    end
	
    begin
        for i from 0:1
            counts[i] = 0
            ready[i] = 0
    end 
\end{verbatim}

\begin{multicols}{2}
\subsection{Hydrogen Processes}
\begin{verbatim}
var ba: BondActivity
ba.incAndCheck(0)
bond()
\end{verbatim}

\subsection{Oxygen Processes}
\begin{verbatim}
var ba: BondActivity
ba.incAndCheck(1)
bond()
\end{verbatim}
\end{multicols}

\subsection{How It Works}
The monitor above makes the assumption that the monitors are
using the "signal and continue" scheme to negotiate which
process gets the monitor upon a signal. However I don't think
that this solution is reliant on that fact, meaning that "signal
and hold" would also work for the above code. In addition I'm
assuming that there is some prewritten function "bond()" that
actually does the bonding action. 

\newpage

\section{Baboon Crossing with CCR's}
\begin{verbatim}
var resources: shared record 
    rCount, lCount, sCount: integer (init = 0)
    inUseL, inUseR, starvingR, starvingL: boolean (init=false)
\end{verbatim}
    
\begin{multicols}{2}
\subsection{Baboon Going Left to Right}
\begin{verbatim}
region resources
do begin
    if(rCount > 0 && inUseL = true) {
        starvingR = true
    }
    lCount ++
end

region resources 
    when(inUseR == false &&
         starvingR == false && 
         sCount < 5) 
begin
    inUseL = true
    sCount++
end

cross()

region resources
begin
    sCount--
    lCount--
    if(sCount == 0) { inUseL = false }
    if(lCount == 0) { starvingL = false }
    endif
end
\end{verbatim}

\subsection{Baboon Going Right to Left}
\begin{verbatim}
region resources
do begin
    if(lCount > 0 && inUseR = true) {
        starvingL = true
    }
    rCount ++
end

region resources 
    when(inUseL == false &&
         starvingL == false && 
         sCount < 5) 
begin
    inUseR = true
    sCount++
end

cross()

region resources
begin
    sCount--
    rCount--
    if(sCount == 0) { inUseR = false }
    if(rCount == 0) { starvingR = false }
    endif
end
\end{verbatim}
\end{multicols}

\subsection{How It Works}
\begin{enumerate}
\item Assumptions: cross() is already implemented properly.

\item First Section: This section makes sure that the opposite
side is not starving. It does this by seeing if the opposite
side's count is greater than 0 and that the current side is
in use. In addition it also increments the count for the
current side.

\item Second Section: This section blocks if the following
conditions are not met. First off the opposide side must not be
in the process of crossing. Also the opposide side must not be
starving, and finally there can be no more than 4 baboons on the
crossing already. If these conditions are met then the current side's
usage is set to true and we increment the count for the number
of baboons currently crossing.

\item Third Section: This section just calls "cross()" an 
assumed function.

\item Fourth Section: This section decrements the count of
baboons on the bridge as well as decrements the count of the 
current side's baboons waiting and/or crossing. Furthermore 
is the current sides count is zero set the current sides usage
flag to false. Finally if there are no current side crossing baboons
set the starving flag to false.
\end{enumerate}
\end{document}

