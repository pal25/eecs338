\documentclass[]{article}

\title{Homework 3}
\author{Patrick Landis (pal25)}
\date{02/21/2013}
\usepackage[margin=1in]{geometry}
\usepackage{multicol}

\begin{document}
\maketitle

\section*{Problem 1}
integer hydroCount = oxyCount = bCount = 0 \\
semaphore mutex(1), oxygen(0), hydrogen(0), barrier(0), bMutex(1)
\begin{multicols}{2}
\textbf{Hydrogen Processes}
\begin{verbatim}
wait(mutex)
hydroCount = hydroCount + 1
if hydroCount >= 2 and oxyCount >= 1:
    hydroCount = hydroCount - 2
    oxyCount = oxyCount - 1
    signal(oxygen)
    signal(hydrogen)
    signal(hydrogen)
else:
    signal(mutex)
endif

wait(hydrogen)

bond()

wait(bMutex)
bCount = bCount + 1
if bCount == 3:
    signal(barrier)
endif
signal(bMutex)

wait(barrier)
signal(barrier)
\end{verbatim}

\textbf{Oxygen Processes}
\begin{verbatim}
wait(mutex)
if hydroCount >= 2:
    hydroCount = hydroCount - 2
    signal(oxygen)
    signal(hydrogen)
    signal(hydrogen)
else:
    oxyCount = oxyCount + 1
    signal(mutex)
endif

wait(oxygen)

bond()

wait(bMutex)
bCount = bCount + 1
if bCount == 3:
    signal(barrier)
endif
signal(bMutex)

wait(barrier)
signal(barrier)
signal(mutex)
\end{verbatim}
\end{multicols}

\textbf{How it works}: \\
Both of the processes check the counts of atoms waiting to bond. If the amount of atoms is sufficient 
then two hydrogen atoms are cleared and one oxygen atom is cleared. "`bond()"' is invoked which is a 
function I'm assuming already exists. This is the first part. Note that the mutex is only signaled
if we do not have enough atoms to bond, otherwise it stays locked so that we know which three atoms
are in the process of bonding while all other atoms must wait at line 1.\\

Before anymore atoms can bond we need to make sure that the three have finished "`bond()"'ing. This is
done by creating another set of semaphores after "`bond()"'ing has occured. Essentially a count is kept
and the bonded atoms can only leave iff all three have bonded. After all three atoms have bonded then
the initial mutex is unlocked so the process can repeat for thre other lucky atoms.
\newpage

\section*{Problem 2}
integer hydroCount = oxyCount = bCount = 0 \\
semaphore mutex(1), oxygen(0), hydrogen(0), barrier(0), bMutex(1)
\begin{multicols}{2}
\textbf{Hydrogen Processes}
\begin{verbatim}
wait(mutex)
hydroCount = hydroCount + 1
if hydroCount >= 2 and oxyCount >= 1:
    hydroCount = hydroCount - 2
    oxyCount = oxyCount - 1
    signal(oxygen)
    signal(hydrogen)
    signal(hydrogen)
else:
    signal(mutex)
endif

wait(hydrogen)

bond()

wait(bMutex)
bCount = bCount + 1
if bCount == 3:
    signal(barrier)
endif
signal(bMutex)

wait(barrier)
signal(barrier)
\end{verbatim}

\textbf{Oxygen Processes}
\begin{verbatim}
wait(mutex)
if hydroCount >= 2:
    hydroCount = hydroCount - 2
    signal(oxygen)
    signal(hydrogen)
    signal(hydrogen)
else:
    oxyCount = oxyCount + 1
    signal(mutex)
endif

wait(oxygen)

bond()

wait(bMutex)
bCount = bCount + 1
if bCount == 3:
    signal(barrier)
endif
signal(bMutex)

wait(barrier)
signal(barrier)
signal(mutex)
\end{verbatim}
\end{multicols}

\textbf{How it works}: \\
Both of the processes check the counts of atoms waiting to bond. If the amount of atoms is sufficient 
then two hydrogen atoms are cleared and one oxygen atom is cleared. "`bond()"' is invoked which is a 
function I'm assuming already exists. This is the first part. Note that the mutex is only signaled
if we do not have enough atoms to bond, otherwise it stays locked so that we know which three atoms
are in the process of bonding while all other atoms must wait at line 1.\\

Before anymore atoms can bond we need to make sure that the three have finished "`bond()"'ing. This is
done by creating another set of semaphores after "`bond()"'ing has occured. Essentially a count is kept
and the bonded atoms can only leave iff all three have bonded. After all three atoms have bonded then
the initial mutex is unlocked so the process can repeat for thre other lucky atoms.
\newpage

\section*{Problem 3}
integer leftCount = rightCount = 0 \\
semaphore mutex(1), blockRight(1), blockLeft(1), leftCross(5), rightCross(5), turnstile(1)
\begin{multicols}{2}
\textbf{Left Entering Baboons}
\begin{verbatim}
wait(turnstile)
wait(blockLeft)
signal(blockLeft)

wait(mutex)
leftCount = leftCount + 1
if leftCount == 1:
    wait(blockRight)
endif
signal(mutex)
signal(turnstile)

wait(leftCross)
cross()
signal(leftCross)

wait(mutex)
leftCount = leftCount - 1
if leftCount == 0:
    signal(blockRight)
signal(mutex)
\end{verbatim}

\textbf{Right Entering Baboons}
\begin{verbatim}
wait(turnstile)
wait(blockRight)
signal(blockRight)

wait(mutex)
rightCount = rightCount + 1
if rightCount == 1:
    wait(blockLeft)
endif
signal(mutex)
signal(turnstile)

wait(rightCross)
cross()
signal(rightCross)

wait(mutex)
rightCount = rightCount - 1
if rightCount == 0:
    signal(blockLeft)
signal(mutex)
\end{verbatim}
\end{multicols}

\textbf{How it works}: \\

The turnstile semaphore is used to ensure that baboons do not starve. Essentially if we have baboons 
going left then, without loss of generality, the turnstile will block once a baboon going right enters
because blockRight will be blocking which is inside the turnstile block. Once this happens anymore
left going baboons will be blocked at the turnstile thus forcing at least one right going baboon to
have a chance of getting to the other side. \\

The 5 max baboons is maintained by leftCross and rightCross, these semaphores will block if there are
5 baboons in the process (pun intended) of crossing. \\

Finally the ability to ensure baboons dont meet while crossing blockLeft anf blockRight are implemented.
What they do is essentially block the opposite going baboons from entering after the first opposite
going baboon has entered the "`cross()"'ing area. We check for the first left/right baboon and block until
there are no more left/right baboons attempting to cross.

\end{document}

